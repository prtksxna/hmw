\biohead{Joan Soutter Lodge}{Joan_Soutter_Lodge}{}

Joan Soutter Lodge was the youngest daughter of Sarah Constance Leake (\p{Sarah_Constance_Leake}) and Thomas Soutter Lodge (\p{Thomas_Soutter_Lodge}).

She married Norman Martin (\p{Norman_Martin}) on 30 April 1913. The report published in \emph{The South-Western News} was as follows:\cite{SouthWestNews1913}

\begin{quotation}
WEDDINGS.

MARTIN--LODGE.

On Tuesday afternoon the marriage took place at St Mary's Church, Busselton, of Mr. Norman Martin, second son of Mr. F. Martin, of ``Lindors,'' St Briavels, Gloucestershire (England), and Miss Joan Soutter Lodge, younger daughter of Mr. and Mrs. Thos. S. Lodge, of ``Strelly,'' Busselton.

The bride, who was given away by her father, wore a beautiful dress of white satin charmeuse, made with a short waist in the style of the Stuart period, with scalloped basque, and belt of pearl embroidery; over bodice of chiffon, magyar sleeves, and tucked vest finished at the neck with and exquisite point lace collar. The draped front of the trained skirt opened over a petticoat of accordion pleated chiffon, and a trail of orange blossoms fell from the waist. The veil was arranged in the new style, one corner being thrown back to form, a cap, and caught on each side of the head with a small bunch of orange blossoms. Her ornaments were a filigree silver necklet and brooch, and blue operculum ear-rings, the gifts of the bridegroom, and she also carried a sheaf a roses, chrysanthemums and ferns.

The bridesmaids were Miss Thelma McLeod, whose frock was a pale blue satin channeuse, with overdress of white crepe ninon, the tunic edged with a fringe of silver beads. The bodice was draped and finished with a pointed lace collar, and long ruched sleeves with net frills falling over the hands, and a folded belt of wine colored satin. Miss Ruth Lukin's frock was of white satin charmeuse, with overdress of bright pink ninon, made with pannier effect; the draped bodice folded over a chemisette of white net, finished at the neck with a big white net raffle, through the centre of which was a band of black velvet. A black velvet belt finished a picturesque frock. Miss Marjorie Clifton, cousin of the bride, wore pink satin charmeuse, with pale grey ninon overdress, and made in the same style. All three brides maids wore Juliet caps of gold net, with posies of pink roses over each ear, and gold shoes. Their dresses were fastened with motor scarf broodies, and they carried white silk parasols, the crook handles being ornamented with bunches of roses and ferns, tied withxibbons to match their dresses, the gifts of the bride groom. Mr. R. J. Lodge, of Roebourne, brother of the bride, supported the bridegroom as best man. The bride's mother was becomingly gowned in violet cryataline, the draped bodice and skirt being trimmed with handsome mauve silk and silver embroidery, and a toque of violet velvet and silk, and shaded uncurled ostrich feathers. Mrs. Aubrey Hall, sister of the bride, wore pink silk, with overdress of pale blue voile, with sprays of sweet peas. The bodice was draped with beautiful point lace. A blade velvet belt, and a large black tagel hat, trimmed with black velvet and shaded roses, and turquoise ornaments completed a charming toilette.

The reception was held at ``Strelly,'' where afternoon tea was served under the picturesque trees at the side of the house. Subsequently the bride and bridegroom left by motor for Yallingup, where they will spend a few days prior to leaving for England by the R.M.S. Orontes\index{Orontes@RMS \emph{Orontes}} on an extended trip. Mrs. Martin travelled in a smart brown and white striped coat and skirt. Her hat was brown felt, turned up with white straw, and trimmed with a brown and white cord, arid the long tail feathers of the Macau.
\end{quotation}
