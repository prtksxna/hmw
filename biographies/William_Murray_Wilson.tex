\biohead{William Murray Wilson}{WilliamMurrayWilson}{}

Murray Wilson (4 June 1914 -- 28 September 1953) was a pilot and timber merchant from Perth, Western Australia. He was murdered in 1953.

Murray was born in \idx{Claremont}\cite{WABMD208} on 4 June 1914 to Jim (\p{James_Herbert_Wilson}) amd Edith (\p{Edith_Olive_Hall}) Wilson.

His father had a house in \idx{Shenton Park} on half an acre.  It was still there, although unrecognisable, in 2005.

Murray served in the RAAF during the \idx{Second World War}. Service number 406376.\cite{NAAA9300} His father died in June 1942, just after Murray turned 28.

For his service during the war, Murray was awarded\cite{RMWDeptAirLetter} four campaign stars (1939--45 Star, Atlantic Star, Africa Star \& Clasp, and Pacific Star) and three medals (Defence Medal, War Medal 1939--45, and Australia Service Medal 1939-45).

%In 1948 C. M. Wilson lent [[Helen Margaret Wilson|Margaret]] one thousand pounds.<ref>Pers. comm. between User:Samwilson and WMW's son Ron.</ref>

By the early `50s, Murray was a director in the timber firm \emph{C M Wilson Co Pty Ltd} which was founded by his [?] and operated from premises on Troode Street in \idx{West Perth}.\cite{CompanyChiefs}

On the evening of Monday, September 28 1953,\cite{WADeaths, WestAusDeathNotices, VictimEstate} just prior to a shareholders' meeting at the company offices, Murray and another director (William Ewart Livingstone) were shot dead by a former director\cite{InsanityVerdict} (and current shareholder) of the company, William Charles Fawcett.  The weapon was a sawn-off .303 rifle,\cite{GuiltyInsane, CompanyChiefs, TimberDirectors} with which Fawcett shot each of his victims once through the heart; they were both dead within minutes.\cite{CompanyChiefs}  Murray worked at the office, and Fawcett had waited outside in his vehicle until he saw Livingstone arrive.\cite{MurderDirectors} They were alone in the office.\cite{NewAngle} He had brought his rifle with him that day with the intention of killing both men.\cite{DoubleMurder}

Fawcett was seen leaving the office after the shooting and driving away, by Livingstone's daughter\cite{FawcettRemanded} He went home to his wife at their home at 66 Victoria Avenue, Claremont (near Murray's home in Congdon Street, Swanbourne)\cite{FawcettRemanded}, told her what he had done, cleaned the gun, and (it is to be supposed) waited for the police to arrive.\cite{DoubleMurder}  He was arrested either one<ref name="S9" /> or four<ref name="smh-double-murder">{{cite news |url=http://nla.gov.au/nla.news-article18379987 |title=Double Murder Charge In W.A. |newspaper=The Sydney Morning Herald |location=NSW |date=30 September 1953 |accessdate=12 March 2014 |page=6 |publisher=National Library of Australia}} "He stood silently at attention during the proceedings, which lasted little more than 30 seconds. He was not represented by counsel."</ref> hours later.  The next day he was charged with murder,<ref name="smh-double-murder" /> but later a jury found against this.<ref name="s4" />

He said that his motive was financial: he believed that the directors were trying to cheat him out of his share in the company\cite{DoubleMurder}<ref name="S14" />.  He had attempted legal action, but didn't have a case.\cite{DoubleMurder}

The trial was held on the evening of Monday, 14 December,<ref name="S14" /><ref name="S5" /> and the ``elderly'', ``diminutive'', ``white-haired''\cite{NewAngle} Fawcett was found not guilty on the grounds of insanity.<ref name="S5" />  He was also in the timber trade.\cite{FawcettRemanded}
