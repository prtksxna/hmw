\biohead{Henry Ernest Hall}{Ernest in about 1905.\cite{ErnestAubreyHannah}}
\index{Hall, Henry Ernest}
\index{Hall, Ernest}

Henry Ernest Hall was born on 7 September 1869 in \idx{Roebourne},\cite{HEHDeathNotice}
the first child of \bioref{William_Shakespeare_Hall} and \bioref{Hannah_Boyd_Lazenby}.

Around 1904 he was managing the \idx{Andover Station} and went to shoot a bullock for meat
but when the animal charged him he managed to shoot himself in the leg and was left lame for life.
The doctor (a Catholic, Dr.~Mansell) charged him \pounds 100 to treat the broken bone,
and despite the high price managed to set it badly.

In about 1916 he moved to Jarmen Island off the coast of \idx{Cossack}
to join his brother \bioref{Harold_Aubrey_Hall} as a lighthouse keeper.\cite{Connie1983}

Ernest was known as the ``strongest man in the nor'-west''.\cite{ErnestHallInscription}

In 1926 he wrote the following letter to the Sunday Times in Perth:\cite{Lepers1926}

\begin{quotation}
LEPERS IN THE NOR'-WEST

Complaint from Cossack

H.E.\ Hall (Cossack) writes:
``I wish to call attention to the manner in which the people of this town are treated by the authorities,
who are not satisfied with bringing the Derby lepers close to Cossack but also allow the white attendant from the leper
station to visit the town and enter the public places, including the telephone box;
I might mention that there is only one medical man for the whites and lepers.''
\end{quotation}

He died of pneumonia,\cite{ErnestHallInscription} aged 71 on 6 June 1941, in Perth.\cite{HEHDeathNotice}
His ashes were interred in the Hall grave in Cossack Cemetery.\cite{ErnestHallInscription}

Death notice:\cite{HEHDeathNotice}
``HALL.—Henry Ernest, elder son of the late William Shakespeare and Hannah Boyd Hall, of Cossack, W.A.,
brother of Aubrey Hall and Joy Clifton: born at Roebourne. W.A.. September 7. 1869. died at Perth. June 6. 1941.''
