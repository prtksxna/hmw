\biohead{Harold Aubrey Hall}{c.\ 1950s\cite{HAHandDarky}}
\index{Hall, Harold Aubrey}
\index{Hall, Aubrey}
\index{Aubrey Hall}

\textbf{Harold Aubrey Hall} (1871--1963) was a Western Australian pastoral station manager and stockman.

He was born in 1871.\cite{ADBWSHall}

Mayor of Cossack from 1898—-1901.\cite{CossackMayoralty}

In March 1904 (aged about 32) he was the sub-agent at \idx{Cossack} for the \idx{Adelaide Steamship Company}.\cite{1904AubreyAgent}

Married \bioref{Helen_Rose_Lodge} in 1910.\cite{HAHmarriage}

In 1912 their first daughter, Connie, was born in Roebourne (in Roe's Cottage).\cite{Connie1966}

In April 1912 was also agent for Henry Wills and Co. in Roebourne.\cite{NorthernTimes1912}

In 1915 Aubrey took a job as lighthouse keeper at Jarmen Island,
the whole family moving there in part to get away from the extreme heat of Croydon.\cite{Connie1983}
Here they shared the accommodation with Mr Langer,
a German national who in 1916 was taken from his post and interned.\cite{Connie1983}
After his departure, Aubrey's brother \bioref{Henry_Ernest_Hall} joined Aubrey on Jarmen Island.

In 1919 their third daughter, \bioref{Joan_Leake_Hall} was born.

From the \emph{Northern Times}, \idx{Carnarvon}, December 1933:\cite{NorthernTimes1933}

\begin{quotation}
A BUSH TRAGEDY.

Whilst travelling in from \idx{Quobba Station} with a thousand sheep for shipment at Carnarvon,
Mr.\ Aubrey Hall, drover, and two natives, Balby and one known as Teddy Edwards,
were accidentally poisoned about 24 miles from Carnarvon.
Hall had arranged for water to be placed at certain places along the route which is a dry one.
About 9 o'clock on Sunday morning whilst travelling along in Colilie paddock on Boolathana
they came across a petrol tin alongside the road which appeared to contain water.
Though they apparently had water in their water bags they drank from the tin, and watered their dogs and horses.
Shortly afterwards they all became sick, the dogs died and also one of the horses,
while later on Edwards be came rapidly worse than the other men and died.
Messages were conveyed to Carnarvon police from Quobba Station whence the news had been carried by a truck driver,
and from Mr.\ Hall who had ridden in to Boolathana Station, following which Constable Summers,
accompanied by Mr.\ George Munro of Dalgety \& Co., left for the scene about 7.30 p.m.\ and
brought the body of Edwards in to Carnarvon morgue.
A post mortem was held by Dr.\ Stewart, but the result is not yet known.
What the tin contained or why it was left alongside the road is at present unknown.
\end{quotation}

He was an adept horseman, and rode well into his 80s. He died in 1963.
