\biohead{William Wilson}{}
\idx{Wilson, William}

William was born in Scotland in 1839, to \bioref{William_Wilson} and \emph{Janet_Garrick}.


| death_date = 1914-05-02
| death_date_precision = day
| death_place = Manly, Sydney
| parent1 = William Wilson snr.
| parent2 = Janet Wilson (née Garrick)
    | partner1 =
    | partner2 =
    | partner3 =
}}
William's formal education ended when he was 18. The family business had failed and it was necessary for the older boys to earn money so that the younger son James could complete his medical training at Edinburgh University. He did this, and was in turn able to educate his own sons, one of them became a leading surgeon in Sheffield, England.<ref name=rae>[[Rae's documents/WILLIA~1.DOC]]</ref>

The Wilsons had been tenant farmers in Ayrshire for many years. They prospered, and moved to Stirling and started milling woollens. This was a good move, but in time there were too many mills for the amount of water available for the waterwheels, and their mill failed.<ref name=rae />

So it was that William, with his older brother [[Robert 'Bob' Crichton Wilson|Robert]] who had already had some trading experience, set off for Melbourne to make their fortunes. Robert soon decided to go to America, where he later started ''The American Trading Company'', trading between America and Australia and various places in the Pacific as well as England. William stayed in Melbourne, trading mainly with Scotland, in drygoods organised by his father. He was never a money-maker, but there is no doubt that he was a well-loved family man. He married in 1864. His wife was [[Mary Wilson (née McHarg)|Mary McHarg]], born in Barrhill in Scotland and had come to Australia with her parents as a child. They were married in North Melbourne.<ref name=rae />

William wrote many letters to his father, one of them somehow came back to Melbourne and [[Lilian Jessie Rae Hussey (née Wilson)|Rae]] had part of it in the 1990s, giving a very good description of Melbourne as it was them. It was probably given to one of William's sons when they visited Scotland.<ref name=rae />

William and Mary had ten children including twin girls one of whom died soon after birth, the other one, [[Jessie Susan Wilson|Jessie]] died as a young girl, of consumption. The others lived well into old age.<ref name=rae />

He died aged 75, on 2 May 1914 at the family house '[[Sturtholm]]' in Manly, Sydney,<ref>"WILSON.— At "Sturtholm," Manly, Sydney, on the 2nd inst., William Wilson, late of Canterbury, Melbourne, and Perth, W.'A., aged 75 years. Burial at Boroondura Cemetery, Kew, Melbourne, on Thursday." — Family Notices (1914, May 5). The Daily Telegraph (Sydney, NSW : 1883 - 1930), p. 6. Retrieved November 11, 2019, from http://nla.gov.au/nla.news-article238809765</ref> and was buried at [[wikipedia:Boroondara General Cemetery|Boroondara General Cemetery]] in Kew, Melbourne.

== References ==
{{smallrefs}}

{{hmw}}
