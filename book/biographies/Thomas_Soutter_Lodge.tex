\biohead{Thomas Soutter Lodge}{}
\index{Lodge, Thomas Soutter}

\emph{The West Australian}, 1 March 1938:\cite{PioneerFarmer1938}

\begin{quotation}
A PIONEER FARMER

The Late Mr. Thomas Lodge.

Mr. Thomas Lodge, who died last Thursday in his 85th year at his residence in Cottesloe,
was long prominently identifled with the fanning industry of the State.
Mr. Lodge was born at Highgate, London, in 1852, and was educated at Clifton College, Gloucestershire, England.
He came to Western Australia in 1878 in a sailing ship, the voyage occupying four months.
From 1878 to 1886 he was settled at Geraldton, with Interests in the North-West.
In 1881 he took a shipment of horses to India for the late Mr. Maitland Brown.
In 1886 he married the fourth daughter of the late Mr. George Walpole Leake, Q.C., thereafter engaging
in farming and stock raising pursuits at Grass Valley, York, Beverley and Busselton.
He was recognised as an expert judge of draught horses and of British breeds of sheep.
Mr. Lodge was keenly interested in all forms of outdoor sports, expecially In cricket and in association
with Mr. R. E. Bush was instrumental in organising several northern cricket teams which visited Perth in the eighties.
Of the scout movement he was an enthusiastic supporter and was probably longer associated with it than anyone else in the State.

Mr. Lodge leaves a widow, three children (Mrs. Aubrey Hall, of Carnarvon,
Mrs. Norman Martin, of Cottesloe, and Mr. Jack Lodge, of Capel) and eleven grandchildren.

His remains were interred in the Church of England portion of the Karrakatta cemetery on Frlday last.
\end{quotation}
