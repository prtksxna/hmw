\biohead{Robert John Lodge, \emph{snr.}}{}

Hampstead and Highgate Express - Saturday 29 December 1888, p7:
\url{https://www.britishnewspaperarchive.co.uk/viewer/bl/0001981/18881229/142/0007}

\begin{quotation}
MR. ROBERT JOHN LODGE.

MR. ROBERT JOHN LODGE, of Highgate, who, since 1839, has been secretary and manager of the Marine Insurance Company,
has retired from work, carrying with him the respect and good wishes of numerous citizens.
He was one of the best-known men in the City, and his genial and merry manner was much appreciated.
He started as a policy clerk in the Alliance Marine,
but his natural ability and indomitable pluck soon raised him to a higher and more lucrative position.
Many good stories are told about his connection with the Marine.
In 1854 insurances were effected in London by a very respectable firm for £40,385, the equivalent of 152,500 Carolus Mexican dollars,
in 61 boxes; for £4,292 on 14,478 ounces Sycee silver; and for £2,400 on flour and bran by the W.T. Sayward from San Francisco to Shanghai.
She was subsequently reported to be lost at the Loo-Choo islands, and a total loss was claimed.
Mr. Lodge, on inquiry, found that the Carolus dollar was a rarity,
and that to procure the quantity said to have been shipped would have been impossible.
This confirmed his suspicions, and, after much trouble, it was found that one Martin Renlok,
a cross between a Greek and a Frenchman, bad been the leader of a conspiracy to defraud the underwriters.
The boxes had been filled with shot and cut nails, and the vessel had been scuttled.
Mr. Lodge detected another carefully-planned fraud in 1859, saving the underwriters nearly £40,000.
On several occasions his colleagues testified, by handsome presents, their sense of the services he had rendered to the craft.
Mr. Lodge passed from the golden age of underwriting to a less fortunate period, when,
on the retirement of Mr. Burnand he had the under' writer's seat. He had the true genius of taking pains,
and it is remarked of him that he could say ``No'' more pleasantly than some men can say ``Yes.''
Duty was his watchword, and be was a man whose word could always he trusted. Being in his seventy-ninth year,
it is not surprising that lie should desire to leave the active sphere of life in which he has been so long engaged.
The estimation iu which he is held by the underwriting community is shown by an address which bears the signatures of
all the leading marine insurance people of London and Liverpool.
The signatories say:---``The ability and persistency with which for half-a-century you have advocated their interests
have long been recognized and appreciated by the underwriting community; while your singleness of purpose and
unfailing courtesy have gained for you the respect and the affectionate regard of all who have been associated with you.''
---City Press.
\end{quotation}

\emph{Norfolk News}, Saturday 15 April 1893, p.13:
\url{https://www.britishnewspaperarchive.co.uk/viewer/bl/0000247/18930415/162/0013}

\begin{quotation}
\textsc{The Late Mr. Robert John Lodge.}---The \emph{Times} gives the following memoir of the late Mr. R. J. Lodge,
the father of Mrs. Henry Morgan, the wife of Mr. Henry Morgan, of the King Street Street Old Brewery.
Our contemporary says:---The death of Mr. Robert John Lodge, for many years the well-known manager of the Marine
Insurance Company, to which office he was appointed in 1839, occurred at his residence at Highgate on the 1st inst.,
within a few days of his 83rd birthday. Mr. Lodge was well known in connection with the salvage operations of the
Royal Charter in 1859, when \textpounds 322,103 was recovered and distributed at a cost of but 5 1.3 per cent.;
also the salvage of \punds 90,000 from the wreck of the Alfonso XII, out of 26 2-3 fathoms of water, an unprecedented
experience. These and similar successes quite revolutionised the premium rate on specie, demonstrating, as they did,
the probability of its recovery even under the most adverse conditions. On the retirement of Mr. Lodge from his
official duties in 1888, he recieved a farewell address signed by the official of some twenty of the principal
marine assurance companies and sixty members of Lloyd's. Mr. Lodge was of a singularly genial temperament, and was
much beloved in Highgate, where he had resided for more than forty years, filling for the last thirteen years the
office of treasurer to the Highgate Literary and Scientific Institution.
\end{quotation}
